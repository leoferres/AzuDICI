
\documentclass{llncs}

\usepackage{graphicx}
\usepackage{url}

\usepackage{amsmath}

\begin{document}

%\conferenceinfo{WXYZ '05}{date, City.} 
%\copyrightyear{2011} 
%\copyrightdata{[to be supplied]} 

%\titlebanner{DRAFT---Do not distribute}        % These are ignored unless
%\preprintfooter{short description of paper}   % 'preprint' option specified.


\newcommand{\barc}{Barelogic$^S$}
\title{Performance of Parallel SAT Solvers}
\titlerunning{Parallel SAT}
\toctitle{Parallel SAT}
%\subtitle{Subtitle Text, if any, or comment out}

\author{Roberto As\'in Ach\'a\inst{1} \and Juan Olate \inst{2} \and Leo Ferres \inst{2}}

\institute{Departamento de Ingenier\'ia Civil Inform\'atica\\
  Facultad de Ingenier\'ia\\ Universidad Cat\'olica de la Sant\'isima
  Concepci\'on, Chile\\ \email{rasin@ucsc.cl} \and Department of
  Computer Science\\Faculty of Engineering\\Universidad de
  Concepci\'on, Chile\\ \email{\{lferres|juanolate\}@udec.cl}}


\maketitle

\begin{abstract}
This is the text of the abstract.
\end{abstract}

\section{Introduction}

This paper discusses the parallel behavior of portfolio-approach
parallel SAT solvers.

SAT...

SAT solvers...

Over the past 35 years, chip manufacturers have relied on three main
areas to achieve performance gains: increasing the amount of work done
per cycle (Out-Of-Order execution), decreasing memory access time
using fast-but-expensive memories (cache-driven hierarchical memory
architectures), and, in particular, increasing the clock cycle count
(more instructions per some unit of time). Up to not too long ago,
about 2003, programs ran faster only in virtue of manufacturers
increasing clock speeds. However, because of several physical problems
(heat, power consumption, and energy leaking), Intel announced in 2004
that 4GHz processors would not be released to the market.  In the
Intel Development Forum of September 2005, Intel announced that there
would be a shift to multicore processors by most major chip
manufacturers, including themselves, given that it was the only way to
keep increasing performance without relying on ever faster
clocks. However, to take advantage of multicore processors, special
software must be developed, and multicore parallel programs have been
hard to write and understand, given their inherent non-determinism, in
particular when sharing memory. Parallel SAT solvers are not the
exception. Although they have been performing at the top of the SAT
competition\footnote{\url{http://www.satcompetition.org/}} (in 2011,
all three wall-clock time winners of the competition are parallel
solvers).

In this paper, we present empirical results that we hope will shed
some light on several aspects of parallel SAT solvers such as memory
consumption, cache contention, and thread communication. We also
suggest solutions to some of the problems.

\section{Barcelogic$^S$}
\label{sec:barsimple}

\section{Cache performance of Barcelogic$^S$}
\label{sec:cachebar}



\appendix
\section{Appendix Title}

This is the text of the appendix, if you need one.

Acknowledgments, if needed.

% We recommend abbrvnat bibliography style.
\bibliographystyle{abbrvnat}

% The bibliography should be embedded for final submission.
\bibliography{bitap}

\end{document}
